CRDT has some features:
\begin{itemize}
	\item The concurrent operations are natively commutative.
	\item The document is a linear sequence of elements.
	\item A single position identifier.
\end{itemize}~

Regarding to experiments, they select different algorithms to generate the single position identifier:
\begin{itemize}
	\item Logoot
	\item RGA
	\item WOOT
	\item WOOTO
	\item WOOTH
\end{itemize}~

The figure \ref{fig:worst} p.\pageref{fig:worst} (with \emph{R} the number of replicas and by \emph{H} the number of operations on the document) shows the theoretical evaluation of these different algorithms. Thus we see that \emph{RGA} and \emph{Logoot} have the best results.

\begin{figure}[h]
  \center
  \includegraphics[width=0.7\textwidth]{includes/worst.png}
  \caption{Worst-case time-complexity analysis}
  \label{fig:worst}
\end{figure}