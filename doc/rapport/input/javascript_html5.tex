\section{Implémentation en \emph{HTML5} avec \emph{Javascript}}
	\label{sec:html5}
  \subsection{Qu'est-ce que \emph{HTML5}}
  
	\emph{HTML} (HyperText Markup Language) est le format de données conçu pour 
	représenter les pages web. 

	\emph{HTML5} est la prochaine révision majeure d'\emph{HTML}. 
	Cette nouvelle version d'\emph{HTML5} est encore en cours de 
	développement et apporte un grand nombre d'améliorations ; de nouvelles 
	balises, de nouveaux attributs, une plus grande rapidité et des	évolutions
	importantes comme les \emph{Web Workers} et les \emph{Events}.
	
	Les \emph{Web Workers} vont permettre de passer outre la limitation 
	historique de Javascript à ne pas supporter les multi-thread.
	Les APIs des \emph{Web Workers} définissent justement un moyen d’exécuter 
	des scripts en tâche de fond. Cela va donc permettre d’exécuter des 
	traitements sur des threads séparés vivant à côté de la page principale et 
	n’ayant pas d’impact sur ses performances d’affichage. 
	
	Les \emph{Server Sent Events} vont permettre d'envoyer des évènements du 
	serveur vers le client, tandis que les \emph{DOM}
	\footnote{Domain Object Model} \emph{Events} vont quand à eux	permettre 
	d'ajouter des évènements à chaque élément d'une page web.
	
	\subsection{Objectif de l'implémentation en \emph{HTML5} avec
		\emph{Javascript}}
	
	L'implémentation avec \emph{Javascript} et \emph{HTML5} permet de réunir 
	l'interface graphique et le modèle du logoot dans un même composant, et non de
	devoir les séparer en deux composants comme en \emph{GWT} (\ref{gwt}), tout en 
	gagnant énormément en rapidité et efficacité.
	
	L'éditeur de texte est une balise éditable grâce à la propriété 
	\emph{contenteditable} et les évènements vont pouvoir être récupérés et 
	ré-implémentés à notre bon vouloir.
	
	\subsection{État d'avancement}

	Pour implémenter notre éditeur, on utilise donc une \emph{div} qui n'est pas
	un élément très contraignant, ce qui permet de récupérer la position du
	curseur de façon assez simple et de parcourir les enfants représentant 
	chaque caractère par des \emph{span} grâces aux méthodes de parcours des 
	éléments du DOM, comme \emph{previousSibbling} ou \emph{parentNode} par 
	exemple.
	
	Concrétement, voilà les fonctionnalités proposés par cette version :
	\begin{itemize}
		\item Insertion et suppression de caractères
		\item Différenciation des clients via couleurs différentes 
	\end{itemize}
	
	L'insertion de caractères (\ref{fig:html5}) va être récupéré et modifié de façon à créer une 
	balise \emph{span} à chaque insertion, contenant en identifiant l'id généré 
	par l'algorithme du logoot, permettant de trier le texte, et d'insérer les 
	caractères directement au bon endroit sans avoir à passer par le calcul des 
	différences pour détecter les modifications éffectués.
	De plus, cela permet d'assigner une couleur à chaque utilisateur 
	en rajoutant un style spécifique à chaque \emph{span} en fonction du
	replica unique à chaque client.

	\begin{figure}[h]
		\label{fig:html5}
		\includegraphics[width=\textwidth]{includes/html5.png}
		\caption{Ajout d'un caractère via l'éditeur \emph{HTML5}}
	\end{figure}
	
	Lors de la suppression, l'évènement est lui aussi récupéré et l'on peut 
	récupèrer directement le \emph{span} supprimer grâce à la position du curseur
	du client, ce qui rend la	suppression très simple, sans avoir à effectuer une
	différence pour trouver les changements effectués.
		
	Pour finir, il reste encore quelques améliorations et finition à porter sur 
	la version \emph{HTML5} du projet :
	\begin{itemize}
		\item Retour à la ligne
		\item Affichage du curseur des différents clients
		\item Suppression multiple de caractères
		\item Copier/coller de caractères
	\end{itemize}
	
	Ces améliorations ne sont pas impossible à rajouter et pourraient être 
	implémentées dans une version future.
	
	Le retour à la ligne doit être représenté avec la balise spécifique 
	\emph{<BR/>} ce qui posait quelques problèmes dans le parcours des éléments du
	DOM lorsque l'on récupérait le parent et les frères suivants des éléments.
	
	L'affichage des curseurs des différents clients s'apparente un peu au problème
	des retours à la ligne dans le sens ou la fonction est prévue en \emph{HTML5} 
	mais est représenté lui aussi par une balise spécifique ce qui imposerait de 
	faire des cas spécifiques en fonction de la valeur du \emph{span} qui 
	pourrait donc contenir soit un caractère soit une balise \emph{HTML}.
	
	La suppression multiple de caractère impose seulement de traiter 
	l'insertion en fonction de la taille de la sélection du curseur, et dans le 
	cas ou la sélection n'est pas vide, traiter l'évènement de manière différente.
	
	De même que la suppression, le copier/coller peut être récupéré via un 
	évènement \emph{HTML5} pré-définis et demanderait donc d'être traité de 
	manière différente que l'insertion d'un simple caractère.
	selection
