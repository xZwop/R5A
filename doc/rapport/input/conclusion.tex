\chapter*{Conclusion}
\addcontentsline{toc}{chapter}{Conclusion}
  L'objectif de ce projet fut d'implémenter l'algorithme \emph{Logoot} pour
  permettre son utilisation concrète dans un éditeur collaboratif sur un réseau
  Pair-à-Pair (à la base des hypothèses de \emph{Logoot} -- causalité, etc.).

  Ce projet à été à l'origine de plusieurs tests. Au départ du projet, nous
  avons effectué trois test de technologies différentes afin de n'en choisir
  qu'une seule. Après plusieurs semaines de développement, il a été décidé
  d'abandonner les versions \emph{Javascript} et \emph{Dart} et de continuer la
  version \emph{Java} \emph{GWT}. Lorsqu'il a fallu intégrer tous les composants
  de l'application, nous nous sommes aperçu que c'était très laborieux. Il
  subsistait aussi le problème de la \emph{textarea} (voir la section
  \ref{sec:textarea} page \pageref{sec:textarea}). Il a donc été décidé de
  reprendre la version \emph{Javascript} et de la coupler à un \emph{div}
  éditable, fournit par \emph{HTML5} (voir la section \ref{sec:html5} page
  \pageref{sec:html5}). Nous avons donc obtenu deux versions fonctionnelles
  possédant toute deux leurs faiblesses. La version \emph{HTML5} avec
  \emph{Javascript} propose une manière de faire très appropriée à
  \emph{Logoot}.

  Ce projet fut très intéressant, du point de vu de l'exploration des
  technologies qu'il a fallu faire, ainsi que pour l'idée principale qui est de
  "concurancer" \emph{Google Docs}.

  Ce projet n'a pas atteint le niveau de finition souhaité. Il faudrait se
  focaliser sur la version \emph{HTML5} avec \emph{Javascript} qui  à la
  particularité d'utiliser l'arbre \emph{DOM} comme modèle de \emph{Logoot}
  (identifiants de ligne comme identifiants des n\oe{}uds \emph{DOM}). Cette
  version ne propose pas encore toutes les fonctionnalités de base d'un éditeur,
  problème qu'il faut résoudre en priorité.

